%---------------------------------------------------
% Nombre: presentacion.tex  
% 
% Texto de presentaci�n del proyecto
%---------------------------------------------------

\mbox{}
\begin{center}
{\large\bfseries An�lisis de tendencias con Big Data}\\
\end{center}

\noindent{\textbf{Palabras clave}: Reglas de Asociaci�n, Big Data, Miner�a de textos, Twitter}\\

{\Large \textbf{Resumen:}}

La miner�a de medios sociales es  uno de los  �mbitos de aplicaci�n de la miner�a de datos m�s estudiados en los �ltimos a�os.  Tanto en el �mbito de empresarial como en el de investigaci�n, estas t�cnicas suscitan un gran inter�s debido a que con el correcto procesamiento pueden obtenerse una gran cantidad de informaci�n y valor de datos que apriori parecen desectruturados. En este trabajo, se propone un sistema basado en miner�a de textos para an�lisis de medios sociales mediante en cual se dar� un flujo de an�lisis de datos en Big Data en Twitter. Esto se conseguir� mediante el an�lisis de patrones, proporcionados por reglas de asociaci�n, cuya utilidad en este �mbito de aplicaci�n quedar� constatada en el exhaustivo estudio del estado del arte llevado a cabo. Se discuten y compran diversas t�cnicas de extracci�n de reglas as� como se evidencian las limitaciones de los algoritmos habituales, los cuales queda demostrada su poca utilidad en problemas enmarcados en el paradigma Big Data. Para poder constatar que los resultados son aceptables o se ajustan a la realidad, el sistema ser� probado con un caso de uso real de Big Data sobre las elecciones generales del Gobierno de Espa�a del 28 de abril, constando el buen funcionamiento del sistema, a pesar de tener m�s de 1.5 millones de transacciones. 

\clearpage
%---------------------------------------------------